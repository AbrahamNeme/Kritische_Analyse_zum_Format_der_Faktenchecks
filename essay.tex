\documentclass[a4paper,listof=totoc,bibliography=totoc]{scrartcl}

\usepackage[ngerman]{babel}
\usepackage[utf8]{inputenc}
\usepackage[T1]{fontenc}
\usepackage{natbib}

\title{Kritische Analyse zum Format der Faktenchecks}
\author{Abraham Neme Alvarez, Hannes Lötsch}
\date{\today}

\begin{document}

\maketitle

\section{Zusammenfassung}

Eine kurze Zusammenfassung der Themen des Aufsatzes sowie der wesentlichen
gewonnenen Erkenntnisse.

\section{Einleitung}

* Die Natur der Fakten: Untersuchung der Kategorien und Eigenschaften von Fakten und der Frage, 
ob sie objektiv und unabhängig von der menschlichen Wahrnehmung oder sozial konstruiert sind und der Interpretation unterliegen.

* Definition und Bedeutung von Faktenchecks

* Bedeutung der kritischen Analyse des Formats von Faktenchecks

Eine Einleitung in das Thema und die Struktur der Arbeit. Dinge sollten
unbedingt zitiert werden. Wenn eine Aussage zitiert wird, kann die Quelle
am Ende stehen \cite{popper:2005}. Man kann jedoch auch schreiben
,,\citet{feyerabend:1986} folgend definieren wir ....".

\section{Rückblick auf die Geschichte der Faktenchecks und ihrer Verbreitung}

* Ursprünge der Faktenchecks

* Entwicklung und zunehmende Popularität von Faktenchecks in den letzten Jahren

* Wichtige Meilensteine und bemerkenswerte Beiträge zur Entwicklung von Faktenchecks

\section{Organisationen und Redaktionen, welche Faktenchecks in Deutschland anfertigen}

* Überblick über prominente Fact-Checking-Organisationen in Deutschland

* Analyse ihrer Methoden und Fact-Checking-Prozesse

* Untersuchung der Rolle von Redaktionen beim Fact-Checking und deren Einfluss

\section{Abwägung von Objektivität und Transparenz beim Faktencheck}

* Diskussion der ethischen Dilemmata, mit denen Fact-Checking-Organisationen konfrontiert sind

* Analyse der potenziellen Verzerrungen bei Faktenchecks und ihrer Auswirkungen auf eine objektive Berichterstattung

* Untersuchung der Bedeutung von Transparenz, Rechenschaftspflicht und Neutralität im Faktenchecking-Prozess

\section{The Impact of Fact Checks on society}

* Die Auswirkungen von Faktenchecks auf die Gesellschaft

* Analyse des Einflusses von Faktenchecks auf das Vertrauen der Öffentlichkeit in Medienunternehmen

* Erforschung der psychologischen und kognitiven Auswirkungen von Faktenchecks auf die Wahrnehmung der Richtigkeit von Nachrichten durch den Einzelnen

* Untersuchung der Rolle von Faktenchecks bei der Gestaltung politischer Narrative und der öffentlichen Meinung in Wahlperioden

* Untersuchung der Strategien, die von Politikern als Reaktion auf Faktenchecks eingesetzt werden

\section{Fehlinformationen: Die Verbreitung von Unwahrheiten}

* Beispiele von Fehlinformationen und deren Konsequenzen

* Untersuchung der Herausforderungen und Strategien bei der Bekämpfung falscher Behauptungen durch Faktenchecks

* Untersuchung der Wirksamkeit von Faktenchecks bei der Bekämpfung von Fehlinformationen

\section{Faktenchecks in der Zeit der sozialen Medien}

* Bewertung der Rolle von Faktencheck im Zeitalter der Social-Media-Plattformen

* Analyse der einzigartigen Herausforderungen und Chancen der Faktenüberprüfung im digitalen Informationsökosystem

\section{Schluss}

* Summary of key findings and insights

* Implications for future research and improvement of fact-checking practices

* Förderung des kritischen Denkens und der Informationsbewertung

* Diskussion der Bedeutung der Aufklärung der Öffentlichkeit über Methoden und Praktiken der Faktenüberprüfung

\bibliography{essay}{}
\bibliographystyle{apalike}

\end{document}
