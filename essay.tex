\documentclass[a4paper,listof=totoc,bibliography=totoc]{scrartcl}

\usepackage[ngerman]{babel}
\usepackage[utf8]{inputenc}
\usepackage[T1]{fontenc}
\usepackage{natbib}

\title{Kritische Analyse zum Format der Faktenchecks}
\author{Abraham Neme Alvarez, Hannes Lötsch}
\date{\today}

\begin{document}

\maketitle

\section{Zusammenfassung}

Eine kurze Zusammenfassung der Themen des Aufsatzes sowie der wesentlichen
gewonnenen Erkenntnisse.

\section{Einleitung}

Eine Einleitung in das Thema und die Struktur der Arbeit. Dinge sollten
unbedingt zitiert werden. Wenn eine Aussage zitiert wird, kann die Quelle
am Ende stehen \cite{popper:2005}. Man kann jedoch auch schreiben
,,\citet{feyerabend:1986} folgend definieren wir ....".

\bibliography{essay}{}
\bibliographystyle{apalike}

\end{document}
